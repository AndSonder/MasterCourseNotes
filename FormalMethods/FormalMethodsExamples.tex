\documentclass[cn, hazy, blue, normal, 12pt]{elegantnote}

\title{形式化方法例题讲解}
\author{Mobyw}
\version{1.0}
\date{\zhtoday}

\usepackage{tikz}
\usepackage{hyperref}
\usepackage{pgfplots}
\usepackage{bookmark}
\usepackage{multirow}
\usepackage{tabularx}
\usepackage[verbose]{xsim}
\usepackage[enableskew,vcentermath]{youngtab}
\usepackage[
    type={CC},
    modifier={by-nc-sa},
    version={4.0},
]{doclicense}

\usetikzlibrary{patterns}
\pgfplotsset{compat=1.18}

\tikzset{
    box/.style ={
            rectangle,              % 矩形节点
            rounded corners = 5pt,  % 圆角
            minimum width   = 50pt, % 最小宽度
            minimum height  = 20pt, % 最小高度
            inner sep = 5pt,        % 文字和边框的距离
            draw=blue               % 边框颜色
        }
}

\begin{document}

\maketitle

% \setlength{\lineskip}{1.5em}
% \setlength{\parskip}{0pt}

\doclicenseThis

本文档为形式化方法各章节的例题,由于部分答案为个人编撰,难免会出现错误,请保证使用 \href{https://github.com/mobyw/MasterCourseNotes/blob/master/FormalMethods/FormalMethodsExamples.tex}{GitHub仓库} 所发布的最新版本. 如遇问题可在 GitHub 上发布 Issue.

\section{命题逻辑}

本章考点:

\begin{enumerate}
    \item 将自然语言描述转换为命题逻辑公式.
    \item 命题逻辑矢列的有效性判断.
    \item 语法分析树的构造与子式的提取.
    \item 使用指派法判断矢列的有效性.
    \item 使用真值表实现到 CNF 的转换.
\end{enumerate}

\begin{exercise}

    Prove the validity of:

    (1) $ (p \wedge q) \wedge r, s \wedge t \vdash q \wedge s $.

    (2) $ p \vdash(p \rightarrow q) \rightarrow q $.

    (3) $ \neg p \rightarrow \neg q \vdash q \rightarrow p $.

\end{exercise}

\begin{solution}[print=true]

    (1) $$
        \begin{array}{lll}
            1 & (p \wedge q) \wedge r & \text { premise }       \\
            2 & s \wedge t            & \text { premise }       \\
            3 & p \wedge q            & \wedge \mathrm{e}_{1} 1 \\
            4 & q                     & \wedge \mathrm{e}_{2} 3 \\
            5 & s                     & \wedge \mathrm{e}_{1} 2 \\
            6 & q \wedge s            & \wedge \mathrm{i} 4,5
        \end{array}
    $$

    (2) $$
        \begin{array}{lll}
            1 & p                               & \text { premise }          \\
            \hline
            2 & p \rightarrow q                 & \text { assumption }       \\
            3 & q                               & \rightarrow \mathrm{e} 2,1 \\
            \hline
            4 & (p \rightarrow q) \rightarrow q & \rightarrow \mathrm{i} 2-3
        \end{array}
    $$

    (3) $$
        \begin{array}{lll}
            1 & \neg p \rightarrow \neg q & \text { premise }          \\
            \hline
            2 & q                         & \text { assumption }       \\
            3 & \neg \neg q               & \neg \neg \mathrm{i} 2     \\
            4 & \neg \neg p               & \text { MT } 1,3           \\
            5 & p                         & \neg \neg \mathrm{e} 4     \\
            \hline
            6 & q \rightarrow p           & \rightarrow \mathrm{i} 2-5
        \end{array}
    $$

\end{solution}


\section{谓词逻辑}

本章考点:

\begin{enumerate}
    \item 将自然语言描述转换为谓词逻辑公式.
    \item 谓词逻辑矢列的有效性判断.
    \item 语法分析树的构造.
    \item 使用指派法判断矢列的有效性.
\end{enumerate}



\begin{exercise}

    利用谓词规范:
    \begin{enumerate}
        \item $B(x,y)$: $x$ 击败 $y$
        \item $F(x)$: $x$ 是一个足球队
        \item $Q(x,y)$: $x$ 是 $y$ 的四分卫
        \item $L(x,y)$: $x$ 输给 $y$
    \end{enumerate}
    和常值符号
    \begin{enumerate}
        \item $c$: 野猫
        \item $j$: 掠夺者
    \end{enumerate}
    把下列句子翻译成谓词逻辑语句:
    \begin{enumerate}
        \item 每个球队都有一名四分卫。
        \item 若掠夺者队击败野猫队,则掠夺者队没有输给每支足球队。
        \item 野猫队击败了一支击败过掠夺者队的球队。
    \end{enumerate}

\end{exercise}

\begin{solution}[print=true]

    \begin{enumerate}
        \item $\forall t~(F(t) \rightarrow \exists m~Q(m, t))$
        \item $B(j,c) \rightarrow \forall t~(F(t) \rightarrow \neg L(j,t))$
        \item $\exists t~(F(t) \wedge B(t, j) \wedge B(c, t))$
    \end{enumerate}

\end{solution}

\begin{exercise}

    证明下面的谓词逻辑公式是有效的:$\exists y (( \forall x P(x)) \rightarrow P(y))$.

\end{exercise}

\begin{solution}[print=true]

    使用推理规则来推导:

    \begin{enumerate}
        \item 假设 $\forall x P(x)$ 为真.
        \item 根据 1,有 $P(y)$ 为真,其中 $y$ 是存在的.
        \item 由于第一步的假设是任意的,因此可以推断 $\forall x P(x) \rightarrow P(y)$ 为真.
        \item 由于存在一个 $y$ 使得 $\forall x P(x) \rightarrow P(y)$ 为真,因此 $\exists y (\forall x P(x) \rightarrow P(y))$ 为真.
    \end{enumerate}

    因此,我们证明了 $\exists y (( \forall x P(x)) \rightarrow P(y))$ 是有效的。简而言之,这个公式表明“如果对于所有 $x$,$P(x)$ 都为真,那么存在一个 $y$ 使得 $P(y)$ 也为真”。这是一个显然的真实际情况,因为只需选择任意一个 $y$,使得 $P(y)$ 为真即可。

\end{solution}

\begin{exercise}

    Prove the walidity of $\forall x P(x) \rightarrow S \vdash \exists x(P(x) \rightarrow S)$.

\end{exercise}

\begin{solution}[print=true]

    为了证明 $\forall x P(x) \rightarrow S \vdash \exists x(P(x) \rightarrow S)$ 的有效性,我们可以采用反证法。

    假设 $\forall x P(x) \rightarrow S$ 是真的,但 $\exists x(P(x) \rightarrow S)$ 是假的。这意味着不存在 $x$ 使得 $P(x) \rightarrow S$ 成立。

    使用 $\exists x(P(x) \rightarrow S)$ 的否定,我们可以写成:

    $\forall x \neg (P(x) \rightarrow S)$

    使用条件语句的逆否命题,我们可以将 $\neg (P(x) \rightarrow S)$ 重写为 $P(x) \land \neg S$:

    $\forall x (P(x) \land \neg S)$

    现在,使用全称实例化规则,我们可以用一个特定的常量替换任何 $x$,比如 $a$,得到:

    $P(a) \land \neg S$

    然而,这与我们的假设 $\forall x P(x) \rightarrow S$ 是矛盾的。由于 $\forall x P(x) \rightarrow S$ 是真的,因此对于任何常量 $a$,$P(a) \rightarrow S$ 都是成立的。因此,$\exists x(P(x) \rightarrow S)$ 必须是真的。

    我们已经证明了如果 $\forall x P(x) \rightarrow S$ 是真的,那么 $\exists x(P(x) \rightarrow S)$ 也必须是真的。因此,原命题 $\forall x P(x) \rightarrow S \vdash \exists x(P(x) \rightarrow S)$ 是有效的。

\end{solution}


\section{时态逻辑}

本章考点:

\begin{enumerate}
    \item CTL 公式的语法分析树的构造.
    \item 给定模型下,LTL 公式路径的选取.
    \item 给定模型下,CTL 公式有效性的判断.
\end{enumerate}

\section{模型检测}

本章考点:

\begin{enumerate}
    \item LTL 公式等价性的证明.
    \item 合式公式
    \item 标记算法
\end{enumerate}

\section{程序验证}

本章考点:

\begin{enumerate}
    \item 证明公式的部分正确性.
    \item 含有 if 和 while 的代码的部分正确性证明.
    \item 证明公式的完全正确性.
\end{enumerate}

\end{document}
