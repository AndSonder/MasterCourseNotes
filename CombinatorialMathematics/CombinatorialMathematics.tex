\documentclass[cn, hazy, blue, normal, 12pt]{elegantnote}

\title{组合数学笔记}
\author{Mobyw}
\institute{Created by Elegant\LaTeX{}}
\version{1.0}
\date{\zhtoday}

\usepackage{tikz}
\usepackage{pgfplots}
\usepackage{bookmark}
\usepackage{multirow}
\usepackage{tabularx}

\pgfplotsset{compat=1.18}

\begin{document}

\maketitle

\section{第一章 排列、组合及二项式定理}

\subsection{基础}

加法规则、乘法规则

排列、组合

集合: $ A=\{a,b,c,d\} $ .

重集: $ B=\{k_1\cdot a_1,k_2\cdot b_2,\cdots,k_n\cdot b_n\} $ .

\subsection{排列问题}

\subsubsection{线排列}

将一些元素排成一条直线.

$ A=\{a_1,a_2,\cdots,a_n\} $ .

$ r $ 是整数,从这 $ n $ 个不同的元素中取出 $ r $ 个按照一定次序排列起来 $ (r\leq n) $ ,称为集合 $ A $ 的 $ r{\text -} $排列,记为 $ P(n,r) $ .

$ A $ 的 $ r{\text -} $排列为 $ A $ 的 $ r{\text -} $有序子集.

\textbf{定理1.1} 对于正整数 $ n,r (r\leq n) $ , 有 $ P(n,r)=n!/(n-r)! $  .

\textit{推论1:} 当 $ 2\leq r \leq n $ 时,有   $ P(n,r)=nP(n-1,r-1) $ .

公式证明直接展开即可,组合分析证明:首先第一个位置上可以从 $ n $ 个元素中选择一个放置,然后剩下的在 $ n-1 $ 个元素中再挑选 $ r-1 $ 个元素.

\textit{推论2:} 当 $ 2\leq r \leq n $ 时,有  $ P(n,r)=rP(n-1,r-1) + P(n-1,r) $ .

公式证明可以利用 \textit{推论1} 得到,组合分析证明大致思路:将 $ r $ 分为两类,一类是必须要包含其中一个元素的情况,另一种情况是一定不包含这个元素的情况. 第一种情况下我们首先在 $ A\backslash \{a_1\} $ 的情况下选取 $ r-1 $ 个元素进行排列,对于上述的所有排列都可以将 $ a_1 $ 放入从而得到所有一定包含 $ a_1 $ 的情况,也就是 $ r\cdot P(n-1,r-1) $ . 第二种情况就是一定不包含 $ a_1 $ 的情况,直接在 $ n-1 $ 个元素中进行 $ r{\text -} $排列即可: $ P(n-1,r) $ .

\textbf{例题:} 9 个字母单词 FRANGMENTS 进行排列,要求字母 A 总是紧跟在 R 的右边,则共有多少排法?

一种简单的思路是将 A,R 看作一个元素,则可以得到: $ P(8,8) $ .

另外一种思路:假设不考虑 A,R 的约束,可以得到 $ P(9,9) $ ,而考虑这个约束, 需要去掉 $ 8*P(8,8) $ .

\subsubsection{圆排列}

一些元素排成一个圆圈的排列

从集合 $ A=\{a_1,a_2,\cdots,a_n\} $ 的 $ n $ 个元素中取出 $ r $ 个元素按照某一种顺序排成一个圆圈,称这样的排列为圆排列(循环排列).

注意:将一个圆排列旋转得到的另一个圆排列视为相同的圆排列.

圆排列的个数为: $ P(n,r)/r=n!/(r(n-r)!) $

\textbf{例题:} 8 个人围成圆桌就餐,一共有多少种就坐方式?如果有两个人不愿意坐在一起又存在多少种就坐方式?

第一个问题是一个简单的圆排列: $ P(8,8)/8 $ .

第二个问题两种思路:

第一种先将无关的 6 个人和另外 2 个人中的一个人拿出来,这 7 个人可以无约束的直接进行圆排列 $ P(7,7)/7 $ ,然后我们就需要考虑,还有一个人,他在坐的时候选择的空间要去掉其中一个人的左右两边的位置,因此对于那个人来说每一次都只有5个位置供他选择,所以写为: $ P(7,7)/7 *5 $ .

第二种思路为先不考虑约束,然后去掉不满足约束的情况:去掉不满足约束的情况可以考虑为先将那两个人绑定(2 种排列),去掉那部分情况即可: $ P(8,8)/8 - 2*P(7,7)/7 $ .

\textbf{例题:} 四男四女圆桌交替就坐方式?

思路:首先明确是一个圆排列问题,其次考虑先将男生安排了,也就是 $ P(4,4)/4 $ ,然后如何将女生插入到其他位置上,第一个女生插入圆桌位置是有4个选择,第二个则有3个选择依次类推,最终得到: $ P(4,4)/4 *4*3*2*1 $ .

\subsubsection{重排列}

前面的线排列和圆排列都是在一个集合 $ A $ 中选出 $ r $ 个元素进行排列,在每一种排列中每一个元素至多出现一次. 现在考虑元素允许出现重复的情况,即考虑重集的情况下: $ B=\{k_1\cdot a_1,k_2\cdot a_2,\cdots k_n\cdot a_n\} $ 中选择 $ r $ 个元素进行排列.

重集 $ B=\{\infty\cdot b_1,\infty\cdot b_2,\cdots ,\infty\cdot b_n\} $ 的 $ r $ 排列的个数为 $ n^r $ .

\textbf{例题:} 由 1,2,3,4,5,6 这六个数字能够组成多少个五位数?又能组成多少个大于 34500 的五位数?

简单思考组成多少个五位数的问题: $ 6^5 $

第二个问题带约束的情况:第一种正向的思路:最高位要大于3,选法只有3种,其他位次随便选: $ 3 * 6^4 $ ;最高位等于3,次高位大于4的情况: $ 2*6^3 $ ;最高位等于3,次高位等于4,次次高位大于5的情况: $ 1 * 6^2 $ ;最后一种情况,高位的前三位分别为3,4,5,之后的两位随便选 $ 6^2 $ . 然后根据加法规则求和. 第二种思路,使用减的方法,无约束条件下的总数为 $ 6^5 $ ,不满足约束的情况有:最高位小于3;最高位等于3,次高位小于4;最高位等于3,次高位等于4,次次高位小于5的情况. 同样可以得到结果.

\textbf{定理:} 重集 $ B=\{n_1\cdot b_1,n_2\cdot b_2,\cdots, n_k\cdot b_k\} $ 的全排列个数为: $ n!/(n_1!\cdot n_2! \cdots n_k!) $ .

\textit{Proof.} 首先将重集中的 $ n_i $ 个 $ b_i $ 分别赋予上标 $ 1,2,\cdots,n_i $ ,即 $ b_i^1,b_i^2,\cdots,b_i^{n_i}(i=1,2,\cdots,k) $ . 将重集 $ B $ 改写为集合 $ A $ ,其中元素个数为 $ n=n_1+n_2+\cdots+n_k $ . 显然集合 $ A $ 的全排列为 $ n! $ . 而对于任意一个 $ b_i $ ,其内部又会有 $ n_i! $ 种排列,类似地可以得到所有的情况. 证毕.

\textbf{例题:} 使用字母 $A,B,C$ 组成五个字母的符号,要求在每一个符号中,$A$ 至多出现 2 次,$B$ 至多出现 1 次,$C$ 至多出现 3 次,求此类符号的个数.

分析这个问题也是一个重排列问题,思考首先在符号中可能出现的情况, $ \{2\cdot A,1\cdot B,2\cdot C\} $ ; $ \{2\cdot A, 0\cdot B, 3\cdot C\} $ ; $ \{1\cdot A,1\cdot B,3\cdot C\} $ . 可能出现的情况只有这三种,分别计算之后使用加法规则即可.

\subsection{组合}

\textbf{定义:} 假设 $ A=\{a_1,a_2,\cdots ,a_n\} $ 是具有 $ n $ 个元素的集合, $ r $ 是一个非负整数,从这 $ n $ 个不同的元素中取 $ r $ 个不考虑次序组合起来 $ r \leq n $ , 称为集合 $ A $ 的 $ r $ 组合,记为 $ C(n,r) $ .

$A$ 的 $r {\text -}$组合是 $A$ 的 $r {\text -}$无序子集.

\textbf{定理:} 对于 $ r\leq n $ 有 $ C(n,r)=P(n,r)/r!=n!/(n-r)!r! $ .

一个 $ r{\text -} $组合是 $r!$ 个 $ r{\text -} $排列; $ C(n,r) $ 个 $ r{\text -} $组合就是 $ r!C(n,r) $ 个 $ r{\text -} $排序.

\textbf{推论1:}  $ C(n,r)=C(n,n-r) $ .

\textbf{推论2:}  $ C(n,r)=C(n-1,r)+C(n-1,r-1) $ . 这也是 Pascal 公式.

对比排列问题下的公式 $ P(n,r)=P(n-1,r)+r\cdot P(n-1,r-1) $

公式推导的方式可以进行直接推导,不多赘述;思考组合分析的方法,这个证明的方法与在证明排列的推论2的情况是类似的,考虑固定一个集合A中的元素,情况变为 $ r $ 个元素中包含 $ a_1 $ ,另一种情况一定不包含 $ a_1 $ .

\textbf{推论3:}  $ C(n,r)=C(n-1,r-1)+C(n-2,r-1)+\cdots +C(r-1,r-1) $ .

推论3可以通过反复使用推论2得到.

\textbf{例题:} 请问数字 510510 可以被多少不同的奇数整除?

首先 $ 510510=2*3*5*7*11*13*17 $ ,除 2 之外共计有 6 个奇数,因此需要整除 510510 一定是除 2 之外的奇素数的积,且每一个积中一个奇数至多出现一次. 那么我们可以得到: $ C(6,1)+C(6,2)+C(6,3)+C(6,4)+C(6,5)+C(6,6)+1=2^6 $ .

\textbf{例题:} 从 1,2,...,1000 中选出三个整数,有多少种选法使得所选的三个整数的和能够被 3 整除?

因为是找的三个数字的和能够被 3 整除,因此要通过考虑余数的情况来考虑存在的所有可能的情况. 因此将所有的数分为三类, $ A=\{1,4,7,\cdots,1000\} $ , 共有 334 个元素; $ B=\{2,5,8,\cdots,998\} $ ,共有 333 个元素; $ C=\{3,6,9,\cdots,999\} $ ,共有 333 个元素. 这三类分别是余数为 1,2,0 的数字的分类. 要使得找三个数能够被 3 整除那么可能的情况就是

(1)三个数来自于同一个集合;
(2)三个数分别来自三个集合.

第一种选法: $ C(334,3)+2\cdot C(333,3) $ ;第二种选法: $ C(334,1)\cdot C(333,1)\cdot C(333,1) $ ,然后根据加法公式得到结果.

\subsubsection{重复组合问题}

从重集 $ B=\{k_1\cdot b_1,k_2\cdot b_2,\cdots ,k_n\cdot b_n\} $ 中选取 $ r $ 个元素不考虑次序组合起来,称为从B中取出 $ r $ 个元素的重复组合. 我们记为 $ F(n,r) $ .

\textbf{定理:} $ B=\{\infty\cdot b_1,\infty\cdot b_2,\cdots,\infty\cdot b_n\} $ 的r-组合数为 $ F(n,r)=C(n+r-1,r) $ .

\textit{Proof.} 假设 $ n $ 个元素 $ b_1,b_2,\cdots,b_n $ 和自然数 $ 1,2,\cdots,n $ 之间是1-1对应的. 于是考虑的任何组合都可以看为一个r个数的组合 $ \{c_1,c_2,\cdots,c_r\} $ .  我们可以认为各个 $ c_i $ 是按照大小次序排列的,相同的 $ c_i $ 连续地排在一起: $ c_1\leq \cdots\leq c_r $ 排列. 令 $ d_i=c_i+i-1,(i=1,2,\cdots,r) $ ,即: $ d_1=c_1,d_2=c_2+1,\cdots,d_r=c_r+r-1 $ . 而我们又知道 $ c_i $ 可以取得的最大值为 $ n $ ,因此 $ d_i $ 可以取得的最大值为 $ n+r-1 $ ,这样就可以得到集合 $ \{1,2,\cdots,n+r-1\} $ 的一个r-组合: $ d_1d_2\cdots d_r,(d_1<d_2<\cdots <d_r) $ ,显然我们会发现对于一种 $ \{c_1,c_2,\cdots,c_r\} $ 的取法便对应一种 $ \{d_1,d_2,\cdots,d_r\} $ 的取法且一定是1-1对应的. 这样看来我们就会发现,对于允许重复地从 $ n $ 个不同元素中取 $ r $ 个元素的组合数和不允许重复地从 $ (n+r-1) $ 个不同元素中取 $ r $ 个元素的组合数是相同的.
$$
    F(n,r)=C(n+r-1,r)
$$
\textbf{例题:} 某一个餐厅有7种不同的菜,为了招待朋友一个顾客需要买14个菜,请问共有多少种买法?

简单分析这就是一个重复组合问题, $ F(7,14)=C(20,14) $ 种买菜的方法. 注意不要想象为排列问题,对于买菜这个事情是没有顺序的说法的.

\textbf{例题:} 求 $ n $ 个无区别的球放入 $ r $ 个有标志的盒子中 $ (n\geq r) $ 而无一空盒的放法.

首先考虑不能为空的问题,因此每个盒子中必须首先放入一个球,下面我们考虑还有剩余的 $ n-r $ 个球应该如何放置的问题. 这时候需要进行分析,因为我们知道因为每一个盒子中再去放入多少个球是没有限制的,因此我们就应该考虑为盒子的集合为重集,球数量为取的元素的数量,也就是 $ F(r,n-r) $ .

\textbf{例题:} 在由数字0,1,2,...,9组成的 $ r $ 位整数所组成的集合中,如果将一个整数重新排列得到另一个整数,则称这两个整数是等价的. 请问:

(1) 有多少个不等价的整数.

(2) 如果数字0,9最多只能出现一次,那么有多少个不等价的整数.

首先第一个问题很简单, $ F(10,r) $ .

第二个问题从两个角度来思考:首先是考虑0和9都不出现的情况,因此重集变为 $ \{\infty \cdot 1,\infty \cdot 2,\cdots,\infty \cdot 8\} $ , 就是8个数的重集的重复组合问题,结果可以写为: $ F(8,r) $ ; 第二种情况是0出现一次或者9出现一次的情况,这两种情况可以看作是等价的,先在9个数的重集中重复组合 $ r-1 $ 个值,然后将0或者9添加进去,结果都表示为: $ F(9,r-1) $ . 最后还有一种情况就是0和9均出现了一次,也就是 $ F(8,r-2) $ ,然后将0,9添加进去,最终使用加法规则可以得到结果为: $ F(8,r)+2\cdot F(9,r-1)+F(8,r-2) $ . 另外还存在一种思路是,无约束的条件下减去不满足条件的情况,也就是: $ F(10,r) $ ,不满足条件的情况分为0大于2次,9大于2次,但是这种情况下我们需要加上一部分,即两个都大于等于2的情况,因为这个情况被剔除了两次,因此我们需要加回来一次,从而最后我们可以计算得到结果为: $ F(10,r)-F(10,r-2)-F(10,r-2)+F(10,r-4) $ .

\textbf{例题:} 求方程 $ x_1+x_2+\cdots+x_n=r $ 的非负整数解的个数. 其中 $ r,n $ 均为正整数.

这个问题相当于是一个重集的r-组合问题,考虑假设 $ b_1,b_2,\cdots,b_n $ 为 $ n $ 个不同的元素,那么重集 $ B=\{\infty \cdot b_1,\infty \cdot b_2,\cdots,\infty\cdot b_n\} $ 的任何一个r-组合都具有 $ \{x_1\cdot b_1,x_2\cdot b_2,\cdots,\cdots,x_n\cdot b_n\} $ ,其中 $ x_i(i=1,2,\cdots,n) $ 是非负整数且满足方程: $ x_1+x_2+\cdots+x_n=r $ . 所以对于这个问题来说,满足上述方程的非负整数解就相当于在这个重集 $ B $ 中进行了一次r-组合的过程,结果也就是 $ F(n,r) $ .

\subsection{二项式定理}

\textbf{定理:} \textit{二项式定理}:当 $ n $ 是一个正整数时,对于任意 $ x $ 和 $ y $ 都有:
$$
    (x+y)^n=\sum_{k=0}^n \binom{n}{k}x^k y^{n-k}
$$
\textit{Proof.} 组合分析法:对于这个问题, $ (x+y)^n $ 表示n个 $ (x+y) $ 的连乘,相当于在 $ n $ 个因子中选择 $ k $ 个,这 $ k $ 个只选择 $ x $ 变量,同时还有 $ n-k $ 个只选择 $ y $ 变量,这样构成了 $ x^ky^{n-k} $ . 此时还要乘上构成这个组合的选法为 $ \binom{n}{k} $ . 第二种证明方法是数学归纳法这里不做详述.

二项式展开式有非常多的变种,下面给出一些:

\textbf{推论1:}当 $ n $ 为正整数时,对于任意的 $ x,y $ 都会有:
$$
    \begin{align}
        (x+y)^n & =\sum_{k=0}^{n}\binom{n}{n-k}x^ky^{n-k}   \\
        (x+y)^n & =\sum_{k=0}^{n}\binom{n}{k}x^{n-k}y^{k}   \\
        (x+y)^n & =\sum_{k=0}^{n}\binom{n}{n-k}x^{n-k}y^{k}
    \end{align}
$$
\textbf{推论2:}在实际情况下常会出现 $ y=1 $ ,因此当 $ n $ 为正整数时,对于所有的 $ x $ 都会有:
$$
    (1+x)^n=\sum_{k=0}^n \binom{n}{k}x^k=\sum_{k=0}^n \binom{n}{n-k}x^k
$$
\textbf{推论3:}当 $ x,y=1 $ 的情况下:
$$
    \sum_{k=0}^n \binom{n}{k}=2^n
$$
\textbf{推论4:}当 $ x=-1,y=1 $ 的情况下:
$$
    \sum_{k=0}^n\binom{n}{k}(-1)^k=0
$$
牛顿在1676年推广了二项式定理,得到了 $ (x+y)^{\alpha} $ 的展开式,其中 $ \alpha $ 为任意的实数. 首先引入记号 $ \binom{\alpha}{k} $ .

\textbf{定义:}对于任意的实数 $ \alpha $ 和整数 $ k $ ,定义:
$$
    \binom{\alpha}{k}=
    \begin{cases}
        \frac{\alpha (\alpha-1)\cdots(\alpha-k+1)}{k!} & k>0 \\
        1                                              & k=0 \\
        0                                              & k<0
    \end{cases}
$$
为广义的\textbf{二项式系数}.

\textbf{定理:}假设 $ \alpha $ 是任意一个实数,则对于满足 $ |\dfrac{x}{y}|<1 $ 的所有 $ x $ 和 $ y $ 有:
$$
    (x+y)^{\alpha}=\sum_{k=0}^{\infty}\binom{\alpha}{k} x^k y^{\alpha-k}
$$

上面的式子也称为牛顿二项式定理. 同样的基于牛顿二项式定理,我们再引入一些推论:

\textbf{推论1:}对于 $ |x|<1 $ 的任何 $ x $ ,有:

$$
    (1+x)^{\alpha}=\sum_{k=0}^{\infty}\binom{\alpha}{k} x^k
$$

\textbf{推论2:}若令 $ \alpha=-n $ ( $ n $ 为正整数),对于 $ |x|<1 $ 的任意 $ x $ 有:

$$
    (1+x)^{-n}=\frac{1}{(1+x)^n}=\sum_{k=0}^{\infty}(-1)^k \binom{n+k-1}{k} x^k
$$

\textbf{推论3:}当 $ |x|<1 $ 时,同时令 $ n=1 $ ,我们可以得到:

$$
    \frac{1}{1+x}=\sum_{k=0}^\infty (-1)^k x^k
$$

\textbf{推论4:}当 $ |x|<1 $ 时,同时令 $ x=-x $ ,我们可以得到:

$$
    \frac{1}{1-x}=\sum_{k=0}^\infty x^k
$$

\textbf{推论5:}使用 $ -rx $ 替代 $ x $ ( $ r $ 为非零常数),当 $ |rx|<1 $ 时有:

$$
    (1-rx)^{-n}=\sum_{k=0}^\infty \binom{n+k-1}{k}r^kx^k
$$

\textbf{推论6:}令 $ \alpha=1/2 $ ,当 $ |x|<1 $ 时有:

$$
    \sqrt{1+x}=1+\sum_{k=1}^{\infty}\frac{(-1)^{k-1}}{2^{2k-1}k}\binom{2k-2}{k-1}x^k
$$

\subsection{组合恒等式}

\textbf{恒等式1:}对于正整数 $ n $ 和 $ k $ ,有:
$$
    \binom{n}{k}=\frac{n}{k}\binom{n-1}{k-1}
$$
\textit{Proof.} 当 $ k>n $ 时, $ \binom{n}{k}=\frac{n}{k}\binom{n-1}{k-1} $ .

当 $ 1\leq k\leq n $ 时,有:
$$
    \begin{aligned}
        \binom{n}{k} & =\frac{n(n-1)\cdots (n-k+1)}{k(k-1)\cdots 1}               \\
                     & =\frac{n}{k}\times \frac{(n-1)(n-2)\cdots (n-k+1)}{(k-1)!} \\
                     & =\frac{n}{k}\binom{n-1}{k-1}
    \end{aligned}
$$
\textbf{恒等式2:}对于正整数 $ n $ ,有:
$$
    \sum_{k=1}^n k\binom{n}{k}=2^{n-1} n
$$

\textit{Proof.}

$$
    \begin{split}
        \sum_{k=1}^n k\binom{n}{k} &= \sum_{k=1}^n k\cdot \frac{n}{k}\binom{n-1}{k-1}\\
        &= \sum_{k=1}^n n \binom{n-1}{k-1}\\
        &= \sum_{k=0}^{n-1} n \binom{n-1}{k}\\
        &= n\sum_{k=0}^{n-1} \binom{n-1}{k}\\
        &= 2^{n-1} n
    \end{split}
$$

\textit{Another Proof.} 使用微分法:

$$
    (1+x)^n =\sum_{k=0}^{n} \binom{n}{k} x^k
$$
两边同时对 $ x $ 做微分得到:
$$
    n(1+x)^{n-1} = \sum_{k=1}^n k \binom{n}{k}x^{k-1}
$$
令 $ x=1 $ 得:
$$
    n\cdot 2^{n-1}=\sum_{k=1}^n k\binom{n}{k}
$$
\textbf{恒等式3:}对于正整数 $ n $ ,有:
$$
    \sum_{k=0}^n(-1)^k k\binom{n}{k}=0
$$
\textit{Proof.}
$$
    (1+x)^n=\sum_{k=0}^n \binom{n}{k}x^k
$$
两边同时对 $ x $ 做微分可以得到:
$$
    n(1+x)^{n-1}=\sum_{k=0}^n \binom{n}{k} k x^{k-1}
$$
在上式中,令 $ x=-1 $ ,则有:
$$
    n\cdot 0 =\sum_{k=0}^n\binom{n}{k}k \cdot (-1)^{k-1}=\sum_{k=0}^n\binom{n}{k} k(-1)^k
$$
\textbf{恒等式4:}对于正整数 $ n $ ,有:
$$
    \sum_{k=0}^n k^2 \binom{n}{k}=2^{n-2}n(n+1)
$$
证明的思路是:首先出发点一定还是二项式定理的推论: $ (1+x)^n=\sum_{k=0}^n \binom{n}{k}x^k $ .首先对于等式两边的式子同时对于 $ x $ 求微分,然后为了构造 $ k^2 $ 我们等式两边同时乘上一个 $ x $ 然后两边同时对 $ x $ 再做一次微分就可以得到结果上面的结果,详细过程不再赘述.

\textbf{恒等式5:}对于正整数 $ n $ ,有:
$$
    \sum_{k=0}^n (-1)^{k-1}\binom{n}{k} k^2=0
$$
\textbf{恒等式6:}对于正整数 $ n $ ,有:
$$
    \sum_{k=0}^n \frac{1}{k+1}\binom{n}{k}=\frac{2^{n+1}-1}{n+1}
$$
\textit{Proof.} 将 $ (1+x)^n=\sum_{k=0}^n\binom{n}{k}x^k $ 两端从0到1积分得到:
$$
    \int_0^1(1+x)^n\mathrm{d}x = \sum_{k=0}^n\binom{n}{k}\int_0^1x^k\mathrm{d}x
$$

$$
    \frac{(1+x)^{n+1}}{n+1}|_0^1=\sum_{k=0}^n\binom{n}{k}\frac{1}{k+1}x^{k+1}|_0^1
$$

令 $ x=1 $ 可以得到:
$$
    \frac{2^{n+1}-1}{n+1}=\sum_{k=0}^n \frac{1}{k+1}\binom{n}{k}
$$
总结:

对于形如 $ \sum k^m\binom{n}{k} $ 的恒等式证明常常需要使用微分法;而对于形如 $ \sum\frac{1}{k+\alpha}\binom{n}{k} $ 的恒等式常常使用积分法来证明,在使用积分证明时一定要注意积分的上下限.

\textbf{恒等式7:}对于正整数 $ n,m $ 和 $ p\leq \min\{m,n\} $ ,有:
$$
    \sum_{k=0}^p\binom{n}{k}\binom{m}{p-k}=\binom{m+n}{p}
$$
这个公式是范德蒙恒等式(Vandermonde恒等式)

\textit{Proof.} 第一种方法使用的是二项式定理:
$$
    \begin{split}
        (1+x)^m(1+x)^n&=(1+x)^{m+n}\\
        \left[\sum_{k=0}^m\binom{m}{k}x^k\right]\left[\sum_{k=0}^n\binom{n}{k}x^k\right]&=\sum_{k=0}^{m+n}\binom{m+n}{k}x^k\\
    \end{split}
$$
比较等式两个 $ x^k $ 的系数,我们可以得到:

$$
    \binom{m}{k}x^k \binom{n}{k}x^k=\binom{m}{k}\binom{n}{k}x^{2k}\\
    \binom{m}{k}x^k \binom{n}{p-k}x^{p-k}=\binom{m}{k}\binom{n}{p-k}x^p
$$

而右边式子中 $ x^p $ 的项为: $ \dbinom{m+n}{p}x^p $ . 而前面的式子中所有的能够得到 $ x^p $ 的项的式子为: $ \sum_{k=0}^p\dbinom{m}{k}\dbinom{n}{p-k}x^p $ .

从而最终我们可以根据系数相等的原则:

$$
    \binom{m+n}{p}=\sum_{k=0}^p\binom{m}{k}\binom{n}{p-k}
$$

\textit{Another Proof.} 第二种方法使用组合分析法:

假设集合 $ A $ 有 $ m $ 个元素, $ B $ 有 $ n $ 个元素,定义 $ A\cup B $ 为集合 $ C $ ,我们假设 $ A\cap B=\empty $ ,那么集合 $ C $ 中就应该有 $ m+n $ 个元素,从这 $ (m+n) $ 个元素中寻找一个p-组合,我们可以得到 $ \binom{m+n}{p} $ . 换一种思路,这个问题就可以变为,我从 $ A $ 中取 $ k $ 个元素,再从 $ B $ 中取 $ p-k $ 个元素的问题,根据乘法规则可以得到: $ \binom{m}{k}\binom{n}{p-k} $ ,同时这种组合的方式根据加法规则,一共有: $ \sum_{k=0}^p\binom{m}{k}\binom{n}{p-k} $ . 同样可以证明恒等式7.

\textbf{恒等式8:}对于正整数 $ m,n $ 有:
$$
    \sum_{k=0}^m\binom{n}{k}\binom{m}{k}=\binom{m+n}{m}
$$
\textbf{恒等式9:}对于任意的正整数 $ n $ 有:
$$
    \sum_{k=0}^m\binom{n}{k}^2=\binom{2n}{n}
$$
\textbf{恒等式10:}对于非负整数 $ p,q,n $ , 有:
$$
    \sum_{k=0}^p\binom{p}{k}\binom{q}{k}\binom{n+k}{p+q}=\binom{n}{p}\binom{n}{q}
$$
\textbf{恒等式11:}对于非负整数 $ p,q,n $ , 有:
$$
    \sum_{k=0}^p\binom{p}{k}\binom{q}{k}\binom{n+p+q-k}{p+q}=\binom{n+p}{p}\binom{n+q}{q}
$$
\textbf{恒等式12:}对于非负整数 $ n $ 和 $ k $ , 有:
$$
    \sum_{i=0}^n\binom{i}{k}=\binom{n+1}{k+1}
$$
可以使用数学归纳法证明.

\textbf{恒等式13:}对于所有实数 $ \alpha $ 和非负整数 $ k $ , 有:
$$
    \sum_{j=0}^k\binom{\alpha+j}{j}=\binom{\alpha+k+1}{k}
$$
使用推广到广义二项式系数的Pascal公式,反复替换可以得到这个恒等式,证明过程略.

\subsubsection{证明组合恒等式的常见方法}

1. 数学归纳法
2. 利用二项式系数公式,特别是Pascal公式
3. 比较级数展开式中的系数(包括二项式定理和母函数法)
4. 积分微分法
5. 组合分析法

\section{离散时间信号与系统}

\subsection{离散时间信号——序列}

\subsubsection{离散时间信号——序列}


\end{document}
